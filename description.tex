\documentclass[12pt]{article}

\usepackage[english]{babel}
\usepackage[utf8x]{inputenc}
\usepackage{amsmath}
\usepackage{graphicx}
\usepackage{longtable}
\usepackage{hyperref}

\usepackage{comment}
\includecomment{todo}
%\excludecomment{changelog}
%\excludecomment{todo}

\newcommand\x         {\hbox{$\times$}}
\newcommand\othername {\hbox{$\dots$}}
\def\eq#1{\begin{equation} #1 \end{equation}}
\def\eqarray#1{\begin{eqnarray} #1 \end{eqnarray}}
\def\eqarraylet#1{\begin{mathletters}\begin{eqnarray} #1
                  \end{eqnarray}\end{mathletters}}
\def\mic              {\hbox{$\mu{\rm m}$}}
\def\about            {\hbox{$\sim$}}
\def\Mo               {\hbox{$M_{\odot}$}}
\def\Lo               {\hbox{$L_{\odot}$}}
\def\comm#1           {{\tt (COMMENT: #1)}}
\def\kms   {\hbox{km s$^{-1}$}}

\usepackage[usenames]{color} 
\newcommand{\G}[1]{{\color{red} #1}}
\newcommand{\B}[1]{{#1}}
\newcommand{\R}[1]{{\color{red}}}
\newcommand{\code}[1]{\texttt{#1}}



\title{Hyper Suprime-Cam Survey \\
  Pipeline Description}
\author{
  The HSC Pipeline Team: \\
  Hisanori Furusawa,
  Michitaro Koike,
  Yuki Okura, \\
  Tadafumi Takata,
  Yoshihiko Yamada (NAOJ Mitaka), \\
  Naoki Yasuda,
  Steve Bickerton,
  Sogo Mineo (IPMU), \\
  Robert Lupton,
  Jim Bosch,
  Craig Loomis, \\
  Hironao Miyatake,
  Paul Price (Princeton) \\
}


\begin{document}
\maketitle
\pagestyle{headings}

\begin{abstract}
\end{abstract}

\clearpage

\tableofcontents

\clearpage

\section{Introduction}

This document describes the Hyper Suprime-Cam Survey Pipeline, which will be used to process the survey
observations and deliver data products with a consistent high quality to the collaboration, and ultimately to
the world, for the realisation of the survey science goals.

\subsection{Hyper Suprime-Cam}

%%%
%%% Stolen from the proposal, with some edits
%%%

Hyper Suprime-Cam takes advantage of the full accessible field of view of the Subaru telescope (1.5~deg$^2$
diameter).  The focal plane is paved with 104 Hamamatsu Deep Depletion science CCDs (plus another 8 guiding
and wavefront monitoring CCDs), each 2k\x 4k. These chips, which are three-side buttable and each have four
independent readout amplifiers, are currently installed in Suprime-Cam, which has demonstrated their excellent
characteristics: low read noise, excellent charge transfer efficiency, few cosmetic defects, and most
importantly, high quantum efficiency from 4000\AA\ to 10,000\AA.  The CCD pixels are $15\mu$m on a side,
corresponding to $0.16$~arcsec at the focal plane. At this resolution, the images will be well-sampled in even
the best seeing.  Ray-tracing of the optics has shown that ghosting is minimal, with the worst ghost at an
illuminance (fractional light in a PSF aperture) of $\sim 5 \times 10^{-8}$.

Details of the instrument are summarized in the Hyper Suprime-Cam
Design Review
Booklet\footnote{\url{http://anela.mtk.nao.ac.jp/hypersuprime/presentation/hscreview20090227final_combined.pdf}}. The
official HSC
webpage\footnote{\url{http://www.naoj.org/Projects/HSC/index.html}}
provides up-to-date information.

\subsection{The HSC Survey}

%%%
%%% Stolen from the proposal, with edits
%%%

The HSC Survey was initiated to address key scientific questions using the new capabilities afforded by HSC.
The principal science drivers are:
\begin{itemize}
\item Studies of the properties of dark matter and dark energy as a
  function of redshift, using measurements of galaxy clustering, weak
  lensing, supernovae, cross-correlation with high-resolution maps
  of the Cosmic Microwave Background, and studies of high-redshift
  quasars.
\item Studies of the evolution of galaxies over cosmic time, with
  emphasis on stellar populations and star formation, morphologies, and
  active nuclei.
\end{itemize}

The survey has a classical ``wedding cake'' design, with three layers.  The Wide layer will cover roughly
1,400~deg$^2$ in five broad filters ($grizy$), with exposure times of $10-20$~min per pointing (and one with a
shorter exposure time of $\sim 30$~sec to enable astrometric and photometric calibration with other surveys).
This will go to a point-source depth of $r\sim 26$, and is optimized for weak-lensing science.  Using galaxies
for weak lensing will require exquisite understanding of their properties and evolution, which will be the
motivation of the Deep layer, covering 27~deg$^2$ in four distinct fields, going roughly one magnitude deeper
with exposures of roughly 1~hr per filter.  In addition to the broad-band filters, the Deep layer will include
imaging in three narrow-band filters, to detect Lyman-$\alpha$ emitters and study their luminosity function
and clustering properties at $z = 2.2$, 5.7, and 6.6.

To study the faintest and most distant galaxies, and to probe the
transient universe, especially high-redshift supernovae, will require
the Ultradeep layer, going another magnitude fainter yet.  Narrow-band
filters in the Ultradeep layer will search for Lyman-$\alpha$
emitters at $z = 5.7$, 6.6, and 7.3 at the very faint end of the
luminosity function.

\subsection{This document}

This document provides a reference to the HSC Pipeline which will be
used to process the HSC Survey observations, producing the necessary
data to achieve the scientific goals of the survey.  It is
especially intended for HSC Survey scientists to evaluate the pipeline
and its products in relation to their own science goals.

In section \ref{sec:pipeline}, we outline the pipeline construction and operations, tracing the flow of data
through the various stages of the pipeline, to the ultimate releases to the collaboration.  Next, in section
\ref{sec:products}, we outline the various products that will compose the data releases, with particular
attention to the measurements that comprise the catalog in the database.  Section \ref{sec:algorithms} gives
details as to the algorithms used to make these measurements, and section \ref{sec:interfaces} describes how
the data will be made available.  Finally, section \ref{sec:support} contains pointers on resources available
for supporting users of these data.



\section{The HSC Pipeline}
\label{sec:pipeline}
\subsection{Software}
\subsection{Data flow}
\subsubsection{Visits}
\subsubsection{Stacks}
\subsubsection{Forced Measurement}
\subsubsection{Multifit}
\subsubsection{Differences}
\subsection{Releases}
\subsubsection{Daily}
\subsubsection{Monthly}
\subsubsection{Semi-annually}

\section{Data Products}
\label{sec:products}
\subsection{Images}
\subsubsection{Visits}
\subsubsection{Stacks}
\subsubsection{Differences}
\subsection{Catalogs}
\subsubsection{Visits}
\subsubsection{Stacks}
\subsubsection{Forced Visit Measurements}
\subsubsection{Forced Stack Measurements}
\subsubsection{Multifit}
\subsubsection{Differences}
\subsection{Database}
\subsubsection{Schema}

\section{Algorithms}
\label{sec:algorithms}
\subsection{Flags}
\subsection{Measurements}
\subsection{Background Matching}
\subsection{CoaddPsf}

\section{External Interfaces}
\label{sec:interfaces}
\subsection{Database}
\subsection{Postage Stamps}
\subsection{Copying}

\section{Help and Support}
\label{sec:support}
\subsection{Forum}
\subsection{Mailing list}


\begin{todo}
\clearpage
\section{TODO}
\begin{itemize}
\item Nothing yet.
\end{itemize}
\end{todo}

\end{document}
